In conclusion, we examined the bode plot, and thereby frequency response of a low pass filter and a band-pass filter.
The low pass filter was measured in the lab using the oscilloscope, using a simple passive component-based RC circuit.


To analyze the band-pass filter, we also included a nyquist plot. The calculated values were very close to the measured values, and the only error source are the passive elements themselves not being entirely ideal,
and the oscilloscope's measurements. The values were as expected, and the cut-off frequencies lined up nicely with what we measured.

Furthermore, we learned how the product of transfer functions can be used to identify more complex filters, such as the band-pass filter, which while
simple in nature is very useful in proving the usefulness of transfer functions.

We found that taking the limit of the low pass filter as the frequency is greater than the cut-off frequency, the amplitude tends to zero, and the phase to -90 degrees, which was observed in our plots as well.
We also found that taking the limit of the low pass filter as the frequency is just at the cut-off frequency, we get a phase shift of about -45 degrees, and as the frequency tends to zero, the amplitude is maximized and the phase is 0 degrees.

The band-pass filter was also built and it was shown to only let a range of frequencies pass through, where an increasing frequency around the center frequency range(about 1000-2000Hz) would result in a larger amplitude, and a decreasing frequency or a larger one would result in a smaller amplitude.