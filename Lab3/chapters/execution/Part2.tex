To determine the properties of a band-pass filter, we had two possible RC combinations.
One combination would lead to a high pass filter, and the other would lead to a low pass filter.

We had two resistors, one of 82K$\Omega$ and one of 10.0K$\Omega$. We also had two capacitors, one of 1.5nF and one of 100nF.

To determine the right combination, we used the formula for the cutoff frequency of a high and low pass filter:
\begin{equation}
    f_c = \frac{1}{2\pi RC}
\end{equation}

And plugging in the values for R=82K$\Omega$ and C=1.5nF, we get:
\begin{equation}
    f_c = \frac{1}{2\pi \cdot 82K\Omega \cdot 1.5nF} = 12939.42\text{Hz}
\end{equation}

Which is the low pass filter, as the cutoff frequency is the greatest out of all the other combinations.


And plugging in the values for R=10K$\Omega$ and C=100nF, we get:
\begin{equation}
    f_c = \frac{1}{2\pi \cdot 10K\Omega \cdot 100nF} = 159.15\text{Hz}
\end{equation}

Which is the high pass filter, as the cutoff frequency is the smallest out of all the other combinations.


We use a sine signal with a 5Vpp amplitude without an offset at the function generator. We vary the frequency of the generator from 50Hz all the way to 100kHz.
We use the oscilloscope to measure the input and output amplitude, and we record the following values:

\begin{adjustwidth}{-2.5 cm}{-2.5 cm}\centering\begin{threeparttable}[!htb]
        \scriptsize
        \begin{tabular}{lrrrrr}\toprule
            \textbf{Input Amplitude(in V)} & \textbf{Output Amplitude(in V)} & \textbf{Phase(in deg)} & \textbf{Frequency(in Hz)} & \textbf{Amplitude} \\\midrule
            10.4                           & 2.92                            & 73.4                   & 50                        & 9.31               \\
            10.4                           & 5.2                             & 53                     & 100                       & 14.32              \\
            10                             & 7.68                            & 38.8                   & 200                       & 17.71              \\
            10                             & 9.44                            & 16.5                   & 500                       & 19.50              \\
            10.4                           & 10                              & 4.32                   & 1000                      & 20.00              \\
            10.9                           & 10.5                            & -4.04                  & 2000                      & 20.42              \\
            11.6                           & 10.5                            & -19.4                  & 5000                      & 20.42              \\
            11.6                           & 9.12                            & -37.7                  & 10000                     & 19.20              \\
            11.6                           & 6.4                             & -55                    & 20000                     & 16.12              \\
            11.6                           & 2.92                            & -73.4                  & 50000                     & 9.31               \\
            12.4                           & 1.54                            & -80.5                  & 100000                    & 3.75               \\
            \bottomrule
        \end{tabular}
        \caption{The effect of the frequency on the output amplitude is shown above. We see what we expect: Between certain frequencies the output amplitude is greater, while when going up to 100kHz or down to 50Hz we notice a significant decline.}
    \end{threeparttable}\end{adjustwidth}